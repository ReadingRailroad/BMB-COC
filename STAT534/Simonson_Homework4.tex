\documentclass[]{article}
\usepackage{lmodern}
\usepackage{amssymb,amsmath}
\usepackage{ifxetex,ifluatex}
\usepackage{fixltx2e} % provides \textsubscript
\ifnum 0\ifxetex 1\fi\ifluatex 1\fi=0 % if pdftex
  \usepackage[T1]{fontenc}
  \usepackage[utf8]{inputenc}
\else % if luatex or xelatex
  \ifxetex
    \usepackage{mathspec}
  \else
    \usepackage{fontspec}
  \fi
  \defaultfontfeatures{Ligatures=TeX,Scale=MatchLowercase}
\fi
% use upquote if available, for straight quotes in verbatim environments
\IfFileExists{upquote.sty}{\usepackage{upquote}}{}
% use microtype if available
\IfFileExists{microtype.sty}{%
\usepackage{microtype}
\UseMicrotypeSet[protrusion]{basicmath} % disable protrusion for tt fonts
}{}
\usepackage[margin=1in]{geometry}
\usepackage{hyperref}
\hypersetup{unicode=true,
            pdftitle={Ecostats Homework 4},
            pdfborder={0 0 0},
            breaklinks=true}
\urlstyle{same}  % don't use monospace font for urls
\usepackage{graphicx,grffile}
\makeatletter
\def\maxwidth{\ifdim\Gin@nat@width>\linewidth\linewidth\else\Gin@nat@width\fi}
\def\maxheight{\ifdim\Gin@nat@height>\textheight\textheight\else\Gin@nat@height\fi}
\makeatother
% Scale images if necessary, so that they will not overflow the page
% margins by default, and it is still possible to overwrite the defaults
% using explicit options in \includegraphics[width, height, ...]{}
\setkeys{Gin}{width=\maxwidth,height=\maxheight,keepaspectratio}
\IfFileExists{parskip.sty}{%
\usepackage{parskip}
}{% else
\setlength{\parindent}{0pt}
\setlength{\parskip}{6pt plus 2pt minus 1pt}
}
\setlength{\emergencystretch}{3em}  % prevent overfull lines
\providecommand{\tightlist}{%
  \setlength{\itemsep}{0pt}\setlength{\parskip}{0pt}}
\setcounter{secnumdepth}{0}
% Redefines (sub)paragraphs to behave more like sections
\ifx\paragraph\undefined\else
\let\oldparagraph\paragraph
\renewcommand{\paragraph}[1]{\oldparagraph{#1}\mbox{}}
\fi
\ifx\subparagraph\undefined\else
\let\oldsubparagraph\subparagraph
\renewcommand{\subparagraph}[1]{\oldsubparagraph{#1}\mbox{}}
\fi

%%% Use protect on footnotes to avoid problems with footnotes in titles
\let\rmarkdownfootnote\footnote%
\def\footnote{\protect\rmarkdownfootnote}

%%% Change title format to be more compact
\usepackage{titling}

% Create subtitle command for use in maketitle
\newcommand{\subtitle}[1]{
  \posttitle{
    \begin{center}\large#1\end{center}
    }
}

\setlength{\droptitle}{-2em}

  \title{Ecostats Homework 4}
    \pretitle{\vspace{\droptitle}\centering\huge}
  \posttitle{\par}
    \author{}
    \preauthor{}\postauthor{}
      \predate{\centering\large\emph}
  \postdate{\par}
    \date{November 18, 2019}


\begin{document}
\maketitle

\section{1) Shrub transition martices}\label{shrub-transition-martices}

\subsection{(a) Asymptotic growth rate}\label{a-asymptotic-growth-rate}

\begin{itemize}
\tightlist
\item
  The asymptoic growth rate, \(\lambda\), is 0.94.
\end{itemize}

\subsection{(b) Stable age
distribution}\label{b-stable-age-distribution}

\begin{itemize}
\tightlist
\item
  The stable age distribution for stages 1-8 are 0.28, 0.03, 0.53, 0.01,
  0.06, 0.06, 0.02, and 0.01, respectively.
\end{itemize}

\subsection{(c) Stable age distribution ignoring
seeds}\label{c-stable-age-distribution-ignoring-seeds}

\begin{itemize}
\tightlist
\item
  The stable age distribution for stages 2-8 (ignoring seeds) is 0.04,
  0.73, 0.02, 0.08, 0.09, 0.02, and 0.01, respectively
\end{itemize}

\subsection{(d) Increasing matrix
elements}\label{d-increasing-matrix-elements}

\begin{itemize}
\tightlist
\item
  The largest element in the sensitivity matrix is row 5, column 3:
  2.90. Since this is a rate(slope), changing that element by 110\%
  (1.1) would result in a change in \(\lambda\) of 3.19.
\end{itemize}

\subsection{(e) Effect of canopy cover}\label{e-effect-of-canopy-cover}

The elements {[}8,1{]} = 237; {[}7,1{]} = 160; {[}8,7{]} = 89 are the
three highest values in the sensitivity matrix.

\subsection{(f) part one}\label{f-part-one}

not even a little close

\subsection{(f) part two}\label{f-part-two}


\end{document}
